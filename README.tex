\documentclass[12pt]{article}
\usepackage{ragged2e} % load the package for justification
\usepackage{hyperref}
\usepackage[utf8]{inputenc}
\usepackage{pgfplots}
\usepackage{tikz}
\usetikzlibrary{fadings}
\usepackage{filecontents}
\usepackage{multirow}
\usepackage{amsmath}
\pgfplotsset{width=10cm,compat=1.17}
\setlength{\parskip}{0.75em} % Set the space between paragraphs
\usepackage{setspace}
\setstretch{1.2} % Adjust the value as per your preference
\usepackage[margin=2cm]{geometry} % Adjust the margin
\setlength{\parindent}{0pt} % Adjust the value for starting paragraph
\usetikzlibrary{arrows.meta}
\usepackage{mdframed}
\usepackage{float}

\usepackage{hyperref}

% to remove the hyperline rectangle
\hypersetup{
	colorlinks=true,
	linkcolor=black,
	urlcolor=blue
}

\usepackage{xcolor}
\usepackage{titlesec}
\usepackage{titletoc}
\usepackage{listings}
\usepackage{tcolorbox}
\usepackage{lipsum} % Example text package
\usepackage{fancyhdr} % Package for customizing headers and footers



% Define the orange color
\definecolor{myorange}{RGB}{255,65,0}
% Define a new color for "cherry" (dark red)
\definecolor{cherry}{RGB}{148,0,25}
\definecolor{codegreen}{rgb}{0,0.6,0}



%%%%%%%%%%%%%%%%%%%%%%%%%%%%%%%%%%%%%%%%%%%%%%%%%%%%%%%%%%%%%%%%%%%%%
% Apply the custom footer to all pages
\pagestyle{fancy}

% Redefine the header format
\fancyhead{}
\fancyhead[R]{\textcolor{orange!80!black}{\itshape\leftmark}}

\fancyhead[L]{\textcolor{black}{\thepage}}


% Redefine the footer format with a line before each footnote
\fancyfoot{}
\fancyfoot[C]{\footnotesize P. Pasandide, McMaster University, Programming for Mechatronics - MECHTRON 2MP3. \footnoterule}

% Redefine the footnote rule
\renewcommand{\footnoterule}{\vspace*{-3pt}\noindent\rule{0.0\columnwidth}{0.4pt}\vspace*{2.6pt}}

% Set the header rule color to orange
\renewcommand{\headrule}{\color{orange!80!black}\hrule width\headwidth height\headrulewidth \vskip-\headrulewidth}

% Set the footer rule color to orange (optional)
\renewcommand{\footrule}{\color{black}\hrule width\headwidth height\headrulewidth \vskip-\headrulewidth}

%%%%%%%%%%%%%%%%%%%%%%%%%%%%%%%%%%%%%%%%%%%%%%%%%%%%%%%%%%%%%%%%%%%%%


% Set the color for the section headings
\titleformat{\section}
{\normalfont\Large\bfseries\color{orange!80!black}}{\thesection}{1em}{}

% Set the color for the subsection headings
\titleformat{\subsection}
{\normalfont\large\bfseries\color{orange!80!black}}{\thesubsection}{1em}{}

% Set the color for the subsubsection headings
\titleformat{\subsubsection}
{\normalfont\normalsize\bfseries\color{orange!80!black}}{\thesubsubsection}{1em}{}


%%%%%%%%%%%%%%%%%%%%%%%%%%%%%%%%%%%%%%%%%%%%%%%%%%%%%%%%%%%%%%%%%%%%%
% Set the color for the table of contents
\titlecontents{section}
[1.5em]{\color{orange!80!black}}
{\contentslabel{1.5em}}
{}{\titlerule*[0.5pc]{.}\contentspage}

% Set the color for the subsections in the table of contents
\titlecontents{subsection}
[3.8em]{\color{orange!80!black}}
{\contentslabel{2.3em}}
{}{\titlerule*[0.5pc]{.}\contentspage}

% Set the color for the subsubsections in the table of contents
\titlecontents{subsubsection}
[6em]{\color{orange!80!black}}
{\contentslabel{3em}}
{}{\titlerule*[0.5pc]{.}\contentspage}


%%%%%%%%%%%%%%%%%%%%%%%%%%%%%%%%%%%%%%%%%%%%%%%%%%%%%%%%%%%%%%%%%%%%%
% set a format for the codes inside a box with C format
\lstset{
	language=C,
	basicstyle=\ttfamily,
	backgroundcolor=\color{blue!5},
	keywordstyle=\color{blue},
	commentstyle=\color{codegreen},
	stringstyle=\color{red},
	showstringspaces=false,
	breaklines=true,
	frame=single,
	rulecolor=\color{lightgray!35}, % Set the color of the frame
	numbers=none,
	numberstyle=\tiny,
	numbersep=5pt,
	tabsize=1,
	morekeywords={include},
	alsoletter={\#},
	otherkeywords={\#}
}




%\input listings.tex



% Define a command for inline code snippets with a colored and rounded box
\newtcbox{\codebox}[1][gray]{on line, boxrule=0.2pt, colback=blue!5, colframe=#1, fontupper=\color{cherry}\ttfamily, arc=2pt, boxsep=0pt, left=2pt, right=2pt, top=3pt, bottom=2pt}




\tikzset{%
	every neuron/.style={
		circle,
		draw,
		minimum size=1cm
	},
	neuron missing/.style={
		draw=none, 
		scale=4,
		text height=0.333cm,
		execute at begin node=\color{black}$\vdots$
	},
}



%%%%%%%%%%%%%%%%%%%%%%%%%%%%%%%%%%%%%%%%%%%%%%%%%%%%%%%%%%%%%%%%%%%%%

% Define a new tcolorbox style with default options
\tcbset{
	myboxstyle/.style={
		colback=orange!10,
		colframe=orange!80!black,
	}
}

% Define a new tcolorbox style with default options to print the output with terminal style


\tcbset{
	myboxstyleTerminal/.style={
		colback=blue!5,
		frame empty, % Set frame to empty to remove the fram
	}
}

\mdfdefinestyle{myboxstyleTerminal1}{
	backgroundcolor=blue!5,
	hidealllines=true, % Remove all lines (frame)
	leftline=false,     % Add a left line
}


\begin{document}
	
	\justifying
	
	\begin{center}
		\textbf{{\large MECHTRON 2MP3 Assignment 2}}
		
		\textbf{Developing a Basic Genetic Optimization Algorithm in C} 
		
		Janahan Sivaneswaran
	\end{center}
	

	
	
	
	\section{Overview}

        \subsection{Included Code}
        \begin{itemize}
		\item \codebox{GA.c}
		\item \codebox{functions.c}
		\item \codebox{Makefile}
	\end{itemize}

        \subsection{Feature Summary}
        \begin{itemize}
		\item Takes user input values for population size, maximum generations, crossover rate, mutation rate, and stop criteria
		\item Lists all inputs and constant values
		\item Computes best fitness values for each generation and continues this operation in a loop until the optimal value is reached
            \item Displays final fitness value and optimal x1 and x2 inputs to reach it
	\end{itemize}
        
	
	
	\section{Makefile Explanation}

	\subsection {Code}
            The following code snippet displays the Makefile included with this report:
		
		\begin{lstlisting}
			CC = gcc
   CFLAGS = -Wall -Wextra -std=c99 
   EXECUTABLE = algorithm
   all: $(EXECUTABLE) 
   $(EXECUTABLE): GA.c
	       $(CC) $(CFLAGS) -o $(EXECUTABLE) GA.c 
   clean:
	       rm -f $(EXECUTABLE)
		\end{lstlisting}

    \subsection{Explanation}
        \begin{itemize}
		\item \codebox{CC = gcc} represents the compiler used to run this program
		\item \codebox{CFLAGS = -Wall -Wextra -std=c99} shows the warning flags and standard used when compiling
		\item \codebox{EXECUTABLE = algorithm} assigns a name to the executable included 
        \item \codebox {all: (EXECUTABLE) } shows the assignment of the "all" target
        \item \codebox {(EXECUTABLE): GA.c} shows that the executable depends on the \codebox {GA.c} file
        \item \codebox {(CC) (CFLAGS) -o (EXECUTABLE) GA.c} is the command line used to compile the code using warning flags
        \item \codebox {clean: rm -f (EXECUTABLE} cleans the workspace after compilation
	\end{itemize}

	\section{Results}
	 
	\begin{table}[h!]
		\caption{Results with Crossover Rate = 0.5 and Mutation Rate = 0.05}
		\label{table:1}
		\centering
		\begin{tabular}{c c c c c c}
			\hline
			Pop Size & Max Gen & \multicolumn{3}{c}{Best Solution} & CPU time (Sec) \\
			& & $x_1$ & $x_2$ & Fitness & \\
			\hline
			10  & 100    & -1.164979  & 0.194164 & 0.080443 & 0.000351\\
			100 & 100    & 0 & 0 & 0.007227 & 0.007391 \\
			1000& 100    & 0 & 0 & 0.000741 & 0.22789\\
			10000& 100    & 0 & -0.338366  & 0.0001 & 17.695008\\
			\hline
			1000  & 1000   & 0 & 0 & 0.000741 & 1.908719\\
			1000 & 10000  & 0 & -1.155303 & 0.000741 & 18.681573\\
			1000& 100000 & 3.680013 & 1.140631  & 0.000741 & 193.417680\\
			1000& 1000000 & -2.377625 & 2.307794  & 0.000741 & 1933.728867\\
			\hline
		\end{tabular}
	\end{table}

	\begin{table}[h!]
		\caption{Results with Crossover Rate = 0.5 and Mutation Rate = 0.2}
		\label{table:1}
		\centering
		\begin{tabular}{c c c c c c}
			\hline
			Pop Size & Max Gen & \multicolumn{3}{c}{Best Solution} & CPU time (Sec) \\
			& & $x_1$ & $x_2$ & Fitness & \\
			\hline
			10  & 100    & 3.309653 & -4.654279 & 0.080443 & 0.000107\\
			100 & 100    & 0 & 4.462964 & 0.007227 & 0.002867\\
			1000& 100    & 0 & 0 & 0.000741 & 0.188489\\
			10000& 100    & -0.348598 & 1.271029  & 0.0001 & 18.377532\\
			\hline
			1000  & 1000   & 0 & 0 & 0.000741 & 1.940264\\
			1000 & 10000  & 2.802171 & 4.144376 & 0.000741 & 19.446068\\
			1000& 100000 & -2.696933 & -0.396226  & 0.000741 & 198.860407\\
			1000& 1000000 & -2.377625 & 2.307794  & 0.000741 & 1933.728867\\
			\hline
		\end{tabular}
	\end{table}
	
	
\end{document}